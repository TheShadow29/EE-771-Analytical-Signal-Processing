\documentclass{article}
\usepackage[a4paper, tmargin=1in, bmargin=1in]{geometry}
\usepackage[utf8]{inputenc}
\usepackage{graphicx}
\usepackage[justification=centering]{caption}

% \usepackage{parskip}
\usepackage{amsmath}
\usepackage{siunitx}
\sisetup{round-mode=places, round-precision=4}
\usepackage{ bbold }

\usepackage{pdflscape}
\usepackage{listings}
\usepackage{hyperref}
\usepackage{caption}
\usepackage{subcaption}
\usepackage{float}

\title{EE 771 : Recent Topics in Analytical Signal Processing Assignment 2}
\author{Arka Sadhu - 140070011}
\date{\today}

\begin{document}
\maketitle

\section*{Q1}
We are given a 1 dimensional periodic signal $x(t)$ with period $T=1$ as:
$$x(t) = a_1 u(t - t_1) + a_2 u(t - t_2) - (a_1 + a_2) u(t - t_3)$$
with $0 < t_1 < t_2 < t_3$
\subsection*{1a}
We want to find the number of samples of $x(t) * sinc(Bt)$ which are sufficient for the reconstrucion of $x(t)$ and a suitable value for $B$. We note that $x(t)$ is an example of a spline and therefore we can directly use Theorem 2 from the paper

\subsection*{1b}
We are now given a new 1d periodic signal $x_1(t)$ with the same period $T=1$ as:
$$x_1(t) = x(t) + b_1 \delta (t - t_1) - b_2 (t - t_3)$$
Clearly, this is an example of the stream of derivatives of dirac deltas and we can use Theorem 3 from the paper.

\section*{Q2}
We want to prove that the Yule Walker system in the algorithm mentioned in the paper is invertible.

Denote the Yule Walker system matrix as $A$. We consider a $3x3$ matrix and note that the proof can be easily extended to any other $nxn$ matrix.

$A = \begin{bmatrix}
X[0] & X[-1] & X[-2]\\
X[1] & X[0] & X[-1]\\
X[2] & X[1] & X[0]
\end{bmatrix}$

Here $X[m] = \frac{1}{\tau} \sum_{k=0}^{K-1} c_k exp(-i 2 \pi m t_k / \tau)$. Denote $u_k = exp(-i 2\pi t_k / \tau)$. We can re-write $X[m] = \frac{1}{\tau} \sum_{k=0}^{K-1}c_k u_k^m$. In this case $K=3$. Also it is clear that the value of $\tau$ wouldn't make a difference in the invertibility of the matrix $A$. Thus we can write the following:
$$X[0] = c_1 + c_2 + c_3$$
$$X[1] = c_1u_1 + c_2 u_2 + c_3 u_3$$
$$X[2] = c_1 u_1^2 + c_2 u_2^2 + c_3u_3^2$$
$$X[-1] = c_1 u_1^{-1} + c_2 u_2^{-1} + c_3 u_3^{-1}$$
$$X[-2] = c_1 u_1^{-2} + c_2 u_2^{-2} + c_3 u_3^{-2}$$

We further note that we can write $A$ as $A = [A_1c | A_2c | A_3c]$. Here:
$$A_3 = \begin{bmatrix}
u_1^{-2} & u_2^{-2} & u_3^{-2}\\
u_1^{-1} & u_1^{-1} & u_3^{-1}\\
 1 &  1 & 1
\end{bmatrix}$$

$$A_2 = UA_3$$
$$A_1 = U^2 A_3$$

$$U = \begin{bmatrix}
u1 & 0 & 0\\
0 & u_2 & 0\\
0 & 0 & u_3
\end{bmatrix}$$

Moreover, $A_3$ is a permutation of a vander monde matrix. Therefore $A_3$ is invertible and therefore is non-singular. Now denote the determinant of $A$ by $det(A)$. We have
$$det(A) = det(A_3 [U^2c | Uc | c]) = det(A_3) det([U^2c | Uc | c])$$
The first term on the rhs is non-zero since $A_3$ is non-singular. The second term on the rhs is
$$det([U^2c | Uc | c]) = c_1 c_2 c_3 det(B)$$
$$B = \begin{bmatrix}
u1^2 & u1 & 1\\
u2^2 & u2 & 1\\
u3^2 & u3 & 1
\end{bmatrix}$$

Clearly B is also a vander monde matrix and therefore, $B$ is also non-singular. Also, $c_1, c_2, c_3$ are also non-zero (else there will be no diracs at those places and the dimension of the matrix will reduce). Therefore we have:
$$det(A) \ne 0$$
Consequently, we have proved that $A$ is invertible.
\section*{Q3}
We are given $u_1, u_2$ as the roots of unity. We want to construct the annhilation filter for the Fourier Series coefficients $$X[m] = \sum_{r=0}^3 c_r m^r u_1^m + \sum_{r=0}^1 d_r m^r u_2^m$$

The annhilation filter $A(z)$ can be constructed in the following way (as noted in the paper).

\section*{Q4}
We are given
$$x(t) = \sum_{k=1}^K b_k \delta (t - t_k)$$
Here all $t_k$ are in $(0, 1)$. Two RC filters with values $(R_1,C_1)$ and $(R_2, C_2)$ are used in parallel to filter out $x(t)$. We are also given the impulse response of the RC filters as:
$$h_i = exp(-R_i C_i) u(t)$$
where $i = 1, 2$ and $u(t)$ is the unit time step signal. We need to find the conditions on sampling interval $T$ so that the sampled signal $x(t) * h_i(t)|_{t = nT}$ are sufficient to reconstruct the parameters of $x(t)$.


\end{document}
